\documentclass[12pt,a4paper]{article}
\usepackage[utf8]{inputenc}
\usepackage[hidelinks]{hyperref}
\usepackage{graphicx}

\renewcommand{\contentsname}{Index}

\title{
    \textbf{Licenciatura em Engenharia Informática e de Computadores} \vspace{5mm} \\
    \textbf{Queuality} \\ 
    Sistema de Gestão de Filas de Espera
}
\author{
    Joana Campos, A44792 \href{mailto:a44792@alunos.isel.pt}{a44792@alunos.isel.pt} 
    \and Nuno Cardeal, A44863 \href{mailto:a44863@alunos.isel.pt}{a44863@alunos.isel.pt}
    \and Carolina Couto, A44871 \href{mailto:a44871@alunos.isel.pt}{a44871@alunos.isel.pt} \\
    \\
    \textbf{Orientador:} Paulo Pereira \href{mailto:paulo.pereira@isel.pt}{paulo.pereira@isel.pt}
}

\begin{document}
\maketitle
\pagebreak
\tableofcontents

\pagebreak
\section{Introdução}
Numa sociedade onde existe cada vez mais adesão a multitasking, estar preso numa fila de espera,
parado sem fazer nada, pode ser considerado uma perda de tempo. Tempo esse que poderia ser usado
para trabalhar ou para “ir beber um café”. Principalmente nestes tempos de pandemia estar em contacto
com outras pessoas deve ser evitado.

Para contornar esta situação, uma solução viável seria um sistema de gestão de filas de espera que,
assumindo que o cliente se encontra próximo do local onde espera ser servido, o notifica quando estiver
próximo da sua vez, evitando assim que o cliente tenha de estar presencialmente na fila, podendo ir
fazer outras coisas enquanto aguarda, melhorando assim a sua “qualidade de vida”.

Neste projeto pretende-se criar um sistema de gestão de filas de espera. O sistema incluirá a
autenticação dos funcionários para que possam realizar a gestão das diferentes filas. Terá ainda de
admitir a existência de pelo menos um administrador que estará encarregue de editar as filas de espera.
Os clientes terão a possibilidade de verem a senha atual de todas as filas, o número de senhas à sua
frente, retirar uma senha ou fazer marcação, e ainda, serão notificados quando a sua vez estiver próxima. 

\pagebreak
\section{Caracterização da Solução}
A solução deste projeto dividir-se-á em quatro partes, uma base de dados que irá guardar a
informação de cada utilizador, uma web API, uma aplicação web dirigida para os funcionários do
serviço e uma aplicação móvel dirigida para os clientes. 

A Web API divide-se em duas partes, a implementação das funcionalidades da aplicação e o
contrato público que disponibiliza as mesmas para as aplicações clientes. Como é possível ver na figura
1 a Web API será o elo de ligação entre as funcionalidades da nossa solução e a aplicação de cliente e
web. A implementação das funcionalidades irá ainda comunicar com a base de dados.

A aplicação web será destinada aos funcionários, sejam eles os que realizam o atendimento aos
clientes ou os que administram o sistema. Estes dois papéis são distintos uma vez que o primeiro estará
encarregue de avançar na fila de espera, e o último terá permissões para alterar o sistema das mesmas.

Para os clientes será disponibilizada uma aplicação móvel que permitirá obter a senha, conhecer
quantas senhas se encontram à sua frente, assim como receber notificações quando faltar um
determinado número de senhas para a sua vez.

\pagebreak
\section{Requisitos Funcionais}
\subsection{Aplicação Web}
\subsection{Aplicação Móvel}

\pagebreak
\section{Requisitos Não Funcionais}
Para a solução deste projeto os recursos serão alocados numa base de dados não relacional, para dar
oportunidade ao aprofundamento dos conhecimentos sobre a mesma. Esta base de dados irá guardar os
utilizadores autenticados e as suas respetivas funções. Cada utilizador anónimo terá uma base de dados
local que guardará marcações e/ou senhas juntamente com a informação adicional que for necessária
no decorrer do projeto.
Tenciona-se realizar deployment na cloud de modo que o projeto tenha oportunidade de ser
distribuído a potenciais interessados. Terá a possibilidade de receber relatórios de análise de forma a
ser possível monitorizar os potenciais problemas que possam vir a ocorrer no sistema.


\end{document}